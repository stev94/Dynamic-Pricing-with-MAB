\begin{abstract}
This work analyses how dynamic pricing can increase the profit in the private parking industry,
by applying some effective bandit algorithms.
This study is carried out in a well defined setting, which is a private parking in the area of Milano.
Three parkings are considered in order to define 3 different classes of users based on the location of their parking.
The first location is Porta Garibaldi, one of the most famous areas of Milano;
the second is Rho Fiera Milano, a neighborhood in the border of the city;
and finally the third is Città Studi, a university neighborhood.
This solution can be extended in a broader scenario, where drivers take advantage of a mobile application in order to
find a parking lot in the city, by booking it and confirm the price per hour in advance.
To simplify our model we assume a monopoly scenario where parking lots are infinite and the cost for selling a parking lot
is assumed as constant and equal to zero without loss of generality.
Our analysis takes also into account the fact that users in different parkings are willing to pay different prices.
We model this fact by assuming a different disaggregate demand curve for each parking, and we compare the performance
of the algorithm that learns the disaggregate demand curves with the one that learns the aggregate one.
In general parkings are characterized by temporal phases and each of them presents a own demand curve.
We identify the specific phases in our scenario and we exploit this information in order to improve the algorithms.
\end{abstract}
