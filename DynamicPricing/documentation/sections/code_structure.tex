\newpage
\section{Code Structure}\label{sec:code-structure}

The code relies on some python libraries: numpy for numerical calculations, matplotlib for plotting functions and scipy both for the calculation of the probability density function for the normal distribution (used for the AB testing) and for the calculation of the demand curves with the pchip function. Pchip is an interpolating function, it's use was very important in the project however because it is able to produce decreasing functions, which are of extreme importance for a demand curve.


The code is structured in a very straight forward way. Below are the most important parts of the code with their roles:
\begin{itemize}
	\item \textbf{Environment class}: Simulates the environment by returning a reward every time an arm is pulled.
	\item \textbf{Data\_Provider class}: A class containing all the information about the input data for our algorithms, meaning the values for all of the demand curves.
	\item \textbf{Learner class}: The super class for the learning algorithms, it stores the results returned by the environment and provides functions for updating all the values based on the observations. It is extended by all the learners, as they all add some more specific functionalities to the class.
	\item \textbf{Plotter class}: A class containing useful functions for drawing plots using the matplotlib library.
	\item \textbf{Target\_Creator class}: Class that generates a curve based on the conversion rates and prices given, and the returns some values from the curvee based on the number of arms we have set. It also calculates which is the optimal arm (with it's price and conversion rate), based on the curve.
	\item \textbf{Sequential\_AB class}: Class containing the sequential AB testing algorithm.
	\item \textbf{Main folder}: Folder that contains multiple main functions, that run a specific scenario of the project and displays the resulting plots. The file 'Main' will run all of the scenarios one by one (it may take some time)
	\item \textbf{Bandit\_algs folder}: Folder that contains all the bandit algorithms. It is partitioned as follows:
	\begin{itemize}
		\item \textbf{Bandit class}: Class that represents a single bandit algorithm.
		\item \textbf{Bandits class}: Class that groups bandit lgorithms based on the selected case (stat, non-stat, context or non-context).
		\item \textbf{Stationary folder}: Folder containing all the relevant code for the non-contextual stationary case. It includes all the relevant learners.
		\item \textbf{Non\_Stationary folder}: Folder contaning all the relevant code for the non-contextual non-stationary case. It uncludes all the relevant learners and a non-stationary environment.
		\item \textbf{Contextual folder}: Folder containing the subclassing of the relevant classes that need to be modified for the contextual case (Bandit, Environment and learner) both in the stationary and non stationary case. 
	\end{itemize}
	
\end{itemize}

