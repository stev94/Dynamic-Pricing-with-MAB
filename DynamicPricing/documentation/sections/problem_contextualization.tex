\newpage
\section{Problem contextualization}\label{sec:problem-contextualization}
\subsection{Use case example}\label{subsec:use-case-example}



A driver is looking for a park and decides to look for it in the private parking of Milano Porta Garibaldi.
When the car is at the entrance, the server will run the algorithms to choose the price per hour for the driver.
The selection of the price will take into account the context of the user, that is the chosen parking, and the phase of the year.
The price is shown to them and they decide if to accept the price and park there or to refuse it and go away.
Finally his decision is sent back to the server in order to update the current state of the algorithms.


\subsection{Classes of users and Phases}\label{subsec:classes-of-users-and-phases}

As stated previously, the users are divided into 3 classes based on the parking lot they chose to use.
This is the simplest and most effective way to identify users based on their features, as other features,
like for example value of their car, age, income, etc. would be difficult to analize in our scenario.
If our system were to be used online, like for example through a mobile application, this would change,
but for now we decided to focus on having the customers buy the parking tickets only at the parking lot.

For each of the parking lots, we have defined a list of four phases through which the parking lots go through.
The phases are all regarding the season, and are cyclical.
Each phase has a different impact on each of the parkings, as they are situated in very different locations,
with different demands throught the year.


\begin{itemize}
	\item Contexts
	\begin{itemize}
		\item Porta Garibaldi
		\item Rho Fiera Milano
		\item Città Studi
	\end{itemize}
	\item Phases
	\begin{itemize}
		\item Winter
		\item Spring
		\item Summer
		\item Autumn
	\end{itemize}
\end{itemize}